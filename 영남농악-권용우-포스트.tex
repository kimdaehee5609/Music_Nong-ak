%	------------------------------------------------------------------------------
%
%		작성 : 2020년 7월 17일 첫 작업
%
%

%	\documentclass[25pt, a1paper]{tikzposter}
%	\documentclass[25pt, a0paper, landscape]{tikzposter}
%	\documentclass[20pt, a1paper ]{tikzposter}
	\documentclass[	20pt, 
							a1paper, 
							portrait, %
							margin=0mm, %
							innermargin=10mm,  		%
							blockverticalspace=4mm, %
							colspace=5mm, 
							subcolspace=0mm
							]{tikzposter}


%	\documentclass[25pt, a1paper]{tikzposter}
%	\documentclass[25pt, a1paper]{tikzposter}
%	\documentclass[25pt, a1paper]{tikzposter}

% 	12pt  14pt 17pt  20pt  25pt
%
%	a0 a1 a2
%
%	landscape  portrait
%

	%% Tikzposter is highly customizable: please see
	%% https://bitbucket.org/surmann/tikzposter/downloads/styleguide.pdf

	%	========================================================== 	Package
		\usepackage{kotex}						% 한글 사용


%% Available themes: see also
%% https://bitbucket.org/surmann/tikzposter/downloads/themes.pdf
%	\usetheme{Default}
%	\usetheme{Rays}
%	\usetheme{Basic}
	\usetheme{Simple}
%	\usetheme{Envelope}
%	\usetheme{Wave}
%	\usetheme{Board}
%	\usetheme{Autumn}
%	\usetheme{Desert}

%% Further changes to the title etc is possible
%	\usetitlestyle{Default}			%
%	\usetitlestyle{Basic}				%
%	\usetitlestyle{Empty}				%
%	\usetitlestyle{Filled}				%
%	\usetitlestyle{Envelope}			%
%	\usetitlestyle{Wave}				%
%	\usetitlestyle{verticalShading}	%


%	\usebackgroundstyle{Default}
%	\usebackgroundstyle{Rays}
%	\usebackgroundstyle{VerticalGradation}
%	\usebackgroundstyle{BottomVerticalGradation}
%	\usebackgroundstyle{Empty}

%	\useblockstyle{Default}
%	\useblockstyle{Basic}
%	\useblockstyle{Minimal}		% 이것은 간단함
%	\useblockstyle{Envelope}		% 
%	\useblockstyle{Corner}		% 사각형
%	\useblockstyle{Slide}			%	띠모양  
	\useblockstyle{TornOut}		% 손그림모양


	\usenotestyle{Default}
%	\usenotestyle{Corner}
%	\usenotestyle{VerticalShading}
%	\usenotestyle{Sticky}

%	\usepackage{fontspec}
%	\setmainfont{FreeSerif}
%	\setsansfont{FreeSans}

%	------------------------------------------------------------------------------ 제목

	\title{ 영남농악 (권용우)  }

	\author{2020년 }

%	\institute{서영엔지니어링}
%	\titlegraphic{\includegraphics[width=7cm]{IMG_1934}}

	%% Optional title graphic
	%\titlegraphic{\includegraphics[width=7cm]{IMG_1934}}
	%% Uncomment to switch off tikzposter footer
	% \tikzposterlatexaffectionproofoff

\begin{document}

	\maketitle[
					width=841mm,
					linewidth = 2mm,
					innersep=4mm,
%					titletotopverticalspace=0mm, %
%					titletoblockverticalspace=0mm, %
					titletextscale =4, 
				]

		%		a0  841 - 1189
		%		a1  594 - 841
		%		a2  420 - 594


	\begin{columns}

		\column{0.5}

%	------------------------------------------------------------------------------
			\block{■  길군악}
			{
				\begin{LARGE}
			\setlength{\leftmargini}{2em}			
			\setlength{\labelsep}{1em} 
				\begin{itemize}
				\item 덩□궁따궁 *2
				\item 덩따궁따궁 *4
				\item 덩□□따궁따궁□덩
				\item 덩따궁따궁따궁□덩
				\item 덩□□덩□□덩
				\end{itemize}

				\end{LARGE}
			}


%	------------------------------------------------------------------------------
		\block{■ 반길군악 }
		{
			\setlength{\leftmargini}{2em}			
			\setlength{\labelsep}{1em} 
			\begin{LARGE}
			\begin{itemize}
			\item 궁따따 / (궁)따따 / 궁따궁 / 따따  * 반복
			\end{itemize}
			\end{LARGE}
		}



%	------------------------------------------------------------------------------
		\block{■ 	다드래기	}
		{
			\setlength{\leftmargini}{2em}			
			\setlength{\labelsep}{1em} 
			\begin{LARGE}
			\begin{itemize}
			\item 궁따따 / (궁)따 * 반복 ( 점점 작게 )
			\end{itemize}
			\end{LARGE}
		}


%	------------------------------------------------------------------------------
		\block{■ 	영산 다드래기	}
		{
			\setlength{\leftmargini}{2em}			
			\setlength{\labelsep}{1em} 
			\begin{LARGE}
			\begin{itemize}
			\item 궁따따 / (궁)따따 * 반복 ( 점점 크게 )
			\end{itemize}
			\end{LARGE}
		}

%	------------------------------------------------------------------------------
		\block{■ 	다드래기	연결 }
		{
			\setlength{\leftmargini}{2em}			
			\setlength{\labelsep}{1em} 
			\begin{LARGE}
			\begin{itemize}
			\item 궁따따 / (궁)따따 / 궁□덩□덩
			\end{itemize}
			\end{LARGE}
		}


%	------------------------------------------------------------------------------
		\block{■ 	다드래기	맺이 }
		{
			\setlength{\leftmargini}{2em}			
			\setlength{\labelsep}{1em} 
			\begin{LARGE}
			\begin{itemize}
			\item 더궁□/더궁□/덩□덩/
			\item 덩□□/□□/덩□덩
			\end{itemize}
			\end{LARGE}
		}

%	------------------------------------------------------------------------------
		\block{■ 	연결채 }
		{
			\setlength{\leftmargini}{2em}			
			\setlength{\labelsep}{1em} 
			\begin{LARGE}
			\begin{itemize}
			\item 덩□/덩□/덩□/(궁따궁따궁) * 2
			\end{itemize}
			\end{LARGE}
		}


%	------------------------------------------------------------------------------
		\block{■ 	별달거리 }
		{
			\setlength{\leftmargini}{2em}			
			\setlength{\labelsep}{1em} 
			\begin{LARGE}
			\begin{itemize}
			\item 덩□덩□/(궁)따궁□ / 궁따(궁)따(궁)따(궁)
			\item 궁따(궁)/궁따(궁) / 궁따(궁)따(궁)따(궁)
			\end{itemize}
			\end{LARGE}
		}




	%	====== ====== ====== ====== ====== 
		\column{0.5}

%	------------------------------------------------------------------------------
		\block{■ 	별달거리2	}
		{

			\setlength{\leftmargini}{2em}			
			\setlength{\labelsep}{1em} 
			\begin{LARGE}
			\begin{itemize}

			\item 하늘보고 별을따고 땅을 보고 농사짓고
			\item 올해도 대풍이요 내년에도 풍년일세
			\item 달아달아 밝은 달아 대낮같이 밝은 달아
			\item 어둠속에 불빛이 우리네를 비취주네

			\end{itemize}
			\end{LARGE}
		}


%	------------------------------------------------------------------------------
		\block{■ 	덧뵈기	}
		{
			\setlength{\leftmargini}{2em}			
			\setlength{\labelsep}{1em} 
			\begin{LARGE}
			\begin{itemize}

			\item 덩 덩 덩 따콩따 *4

			\end{itemize}
			\end{LARGE}
		}

%	------------------------------------------------------------------------------
		\block{■ 	벽구놀이	}
		{
			\setlength{\leftmargini}{2em}			
			\setlength{\labelsep}{1em} 
			\begin{LARGE}
			\begin{itemize}

			\item 덩 따 따콩 따콩따
			\item 덩 따콩 따콩 따콩따

			\end{itemize}
			\end{LARGE}
		}


%	------------------------------------------------------------------------------
		\block{■ 	쌍진풀이	}
		{
			\setlength{\leftmargini}{2em}			
			\setlength{\labelsep}{1em} 
			\begin{LARGE}
			\begin{itemize}

			\item 쿵따따콩따따쿵따따콩따따 
			\item 쿵따따콩따따쿵따따콩따 *4
			\item 쿵따따콩따 *4
			\item 쿵따따콩따따쿵따따콩따따 
			\item 쿵따따콩따따쿵 덩 덩
			\item 덩쿵 덩쿵 덩 덩
			\item 덩

			\end{itemize}
			\end{LARGE}
		}

%	------------------------------------------------------------------------------
		\block{■ 	쌍진풀이	}
		{
			\setlength{\leftmargini}{2em}			
			\setlength{\labelsep}{1em} 
			\begin{LARGE}
			\begin{itemize}

			\item 쿵따따콩따따쿵따따콩따따 
			\item 쿵따따콩따따쿵따따콩따 *4
			\item 쿵따따콩따 *4
			\item 쿵따따콩따따쿵따따콩따따 
			\item 쿵따따콩따따쿵 덩 덩
			\item 덩쿵 덩쿵 덩 덩
			\item 덩

			\end{itemize}
			\end{LARGE}
		}


	\end{columns}




\end{document}


		\begin{huge}
		\end{huge}

		\begin{LARGE}
		\end{LARGE}

		\begin{Large}
		\end{Large}

		\begin{large}
		\end{large}

